\section*{Discussion}

The widespread belief that Turing systems generate stable Turing patterns has dominated both the mathematical and biological investigations into Turing pattern formation. Contrary to this, we have shown that being a Turing system is neither a necessary nor sufficient condition for being a stable pattern former. On the one hand, a Turing analysis of a system cannot tell us whether a given system will form stable patterns, only whether it will form potentially-unstable incipient patterns; on the other hand, the systems capable of pattern inception are the generalised Turing systems, a much larger class than the classical Turing systems. Any pattern inceptor could prove to be a pattern former if the ingredients are present for pattern stabilisation -- there is nothing special about the Turing systems in this regard, and plenty of other systems are capable of pattern stabilisation. However, there may be no need for a pattern to ultimately stabilise, if its role is simply to signal the spatially-differentiated intiation of a developmental process. 

There is, however, one respect in which the Turing systems are special: they can be thought of as robust pattern inceptors. Fluke initial concentrations are not a serious risk to Turing systems, while they are a risk to a small minority of all other types of generalised Turing systems. Furthermore, the entire set of generalised Turing systems is substantially larger than that of the classical Turing systems, implying that they are significantly more robust to parameter changes. A parameter change which makes a Turing system into a mere generalised Turing system would previously have been considered disastrous: we can now see that such a change is unlikely to make much difference to the pattern-forming capabilities of the system.

There are two main consequences of our work for biologists, whether trying to understand developmental processes, or engineer synthetic patterns. First, the conditions for being a Turing system (such as activation-inhibition) are not sufficient to determine the formation of stable patterns: the only guaranteed method is to numerically solve the RDEs describing the system. Second, the set of candidate networks for pattern formation is much larger than is usually supposed: it is essential to look beyond only AI and AS networks (or equivalent lists for three and four species systems \cite{marcon2016high}) when seeking to explain or engineer biological patterns. The development of new analysis approaches is likely to further facilitate these endeavours, including techniques for network enumeration \cite{marcon2016high}, and simplifying analysis via model reduction \cite{smith2018model} or through exploiting conservation laws \cite{halatek2018rethinking}.

Over all, the work in this article suggests that potential pattern-forming systems could be substantially more common and more robust than is typically supposed, and concomitantly potentially easier to discover in living organisms, and easier to engineer in living cells.