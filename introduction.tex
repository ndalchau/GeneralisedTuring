\section*{Introduction}

The Turing mechanism outlines how the amplification of resonant frequencies in reaction-diffusion systems can result in the formation of chemical standing waves, commonly known as Turing patterns \cite{turing1952chemical}. Turing hypothesised that his mechanism could explain biological patterns such as the regular spacing between hydra tentacles, and while the physical plausibility of Turing's model was initially doubted in favour of alternatives like the French Flag model \cite{wolpert1969positional}, he was ultimately proved correct by the discovery of the CIMA reaction \cite{castets1990experimental} and later the WNT/DKK genes controlling mouse hair follicle spacing \cite{sick2006wnt,maini2006turing}.

There is little doubt today that the Turing mechanism underpins a wide variety of biological patterns from zebrafish skin \cite{nakamasu2009interactions} to mouse interdigital spacing \cite{sheth2012hox}, but it is proving difficult to isolate precisely which diffusing chemicals and specific interactions are involved in the underlying patterning network \cite{marcon2012turing}. A seminal paper by Gierer and Meinhardt suggested that Turing patterns could be generated by a system composed of just two diffusing chemical species: a short-range activator and a long-range inhibitor \cite{gierer1972theory}. So-called activator-inhibitor (AI) networks have exerted a powerful hold over the search for real-life Turing systems, and candidate systems to explain biological patterns are almost universally of the AI form \cite{kondo2010reaction}.

There is no reason in principle why Turing systems ought to be AI systems \cite{halatek2018self}. The classical mathematical definition of a Turing system -- derived by Othmer and Scriven \cite{othmer1969interactions} and later formalised by Segel and Jackson in the case of two-species systems \cite{segel1972dissipative}, and Murray more generally \cite{murray1977lectures} -- is based on the particular structure of a function of the reaction rates and diffusion coefficients known as the dispersion relation: if the dispersion relation has the correct form, then the system is deemed to be a Turing system. In a system of two chemical species, the conditions on the dispersion relation imply the existence of exactly two possible patterning network types: the AI network, and the activator-substrate (AS) network \cite{murray2001mathematical}.

When systems of three or more species are considered, the variety of plausible Turing networks increases dramatically \cite{klika2012influence,marcon2016high}. The comparative analytical intractibility of these larger networks meant that little was known about them, until all possible types of 3 and 4 species Turing networks were enumerated in a computational study by Marcon et al. \cite{marcon2016high}. The study revealed that several widely-held assumptions about the kinds of systems that can generate Turing patterns (such as the requirement for diffusing species to have different diffusion coefficients) in fact applied only to two species systems, and implied that the set of potential Turing patterning systems is much more diverse than is commonly supposed.

The work by Marcon et al. was timely, and coincided with a surge in interest in engineering synthetic Turing patterns in living cells \cite{diambra2014cooperativity,borek2016turing,scholes2017three,Karig2018}. 
Such engineered systems typically consist of dozens of explicitly modelled species \cite{smith2018model}, and so an understanding of Turing patterning beyond the two-species case is essential. 
Despite considerable effort to engineer synthetic Turing patterns in a cellular system, only a single example has appeared \cite{Karig2018}. 
The reasons for slow progress are diverse, and principally believed to relate to the robustness problem: the fact that a small change in parameter values can render a patterning system useless \cite{crampin1999reaction,maini2012turing,Scholes2018}. 
And yet the fact that Turing patterns exist in nature implies that robustness cannot be an insuperable barrier.
One important is the role of stochasticity, which has been shown to expand the region of pattern-forming parameter sets \cite{biancalani2010stochastic,butler2011fluctuation}, and is thought to be important for pattern formation in the synthetic biology example of Karig et al. \cite{Karig2018}, but also in the development of the multicellular cyanobacterial species \emph{Anabaena sp.} PCC 7120 \cite{dipatti2018robust}.

In this article, we suggest that the twin difficulties of understanding and engineering biological patterns could be partially due to a completely different reason from the robustness problem. 
We show that the mathematical conditions of \emph{diffusion-driven instability}, while a convenient framework for analysing pattern formation, are neither necessary nor sufficient for the formation of stable patterns via the Turing mechanism. There exist reaction-diffusion systems satisfying all conditions (including AI systems) that do not form long-lasting patterns; and there exist systems which violate several of the conditions, but nonetheless generate stable patterns. We are not talking of systems that form patterns via a different mechanism (such as French Flag \cite{wolpert1969positional}, or oscillating Turing-Hopf systems \cite{baurmann2007instabilities}), but rather systems that form stable patterns via the Turing mechanism, but are nonetheless excluded from the consensus definition of Turing systems. There also systems that satisfy the consensus Turing instability conditions, but form patterns that coarsen over time \cite{kolokolnikov2006}, and ultimately adopt a stable pattern that is not well-predicted by the dispersion relation alone.

The question of which systems will ultimately form stable patterns is a highly nonlinear one, and largely beyond the more routine application of linear mathematics. 
And indeed, there are now numerous nonlinear analysis methods for Turing instabilities, which can overcome the danger of performing only linear analyses \cite{}, but their application is more challenging, which explains why dispersion relations arising from linear stability analysis are more commonplace. 
Nevertheless, here we reframe the Turing patterning question from one of pattern formation to one of pattern \emph{inception}. 
In other words, we ask whether a system is capable of forming a pattern in the first place, irrespective of whether it is (locally) stable. 
We find that the conditions on the dispersion relation for pattern are considerably more general than the consensus conditions, and allow for a much greater variety of patterning networks. 
Systems satistfying these conditions -- which we call ``generalised Turing systems'' -- include two-species activator-activator (AA) networks, and networks consisting of a long-range activator and short-range inhibitor. 

Our results imply that a wide range of plausible patterning networks may have been overlooked as candidate systems, both to explain observed biological phenonmena, and as potential synthetic Turing networks. Our results further suggest that the robustness problem may be less significant than is usually thought, since the class of generalised Turing systems is considerably larger than that of consensus Turing systems, and consequently less sensitive to small parameter changes. As a result, our work opens the door to easier discovery of the biochemical basis of Turing patterns in biology, and easier engineering of synthetic Turing networks in living cells.
