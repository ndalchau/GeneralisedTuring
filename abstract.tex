\begin{abstract}
The Turing patterning mechanism is believed to underly the formation of repetitive structures in development, such as zebrafish stripes and mammalian digits, but it has proved difficult to isolate the specific biochemical species responsible for pattern formation. Meanwhile, synthetic biologists have designed Turing systems for implementation in cell colonies, but none have yet led to visible patterns in the laboratory. In both cases, the relationship between underlying chemistry and emergent biology remains mysterious. To help resolve the mystery, this article asks the question: what kinds of biochemical systems can generate Turing patterns? We find general conditions for Turing pattern inception -- the ability to generate unstable patterns from random noise -- which may lead to the ultimate formation of stable patterns, depending on biochemical non-linearities. We find that a wide variety of systems can generate stable Turing patterns, including several which are currently unknown, such as two-species systems composed of two self-activators, and systems composed of a short-range inhibitor and a long-range activator. We furthermore find that systems which are widely believed to generate stable patterns may in fact only generate unstable patterns, which ultimately converge to spatially-homogeneous concentrations. Our results suggest that a much wider variety of systems than is commonly believed could be responsible for observed patterns in development, or could be good candidates for synthetic patterning networks. 
\end{abstract}